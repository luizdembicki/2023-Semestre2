\documentclass[]{article} % fonte 12, frente e verso, a4
\usepackage{xcolor}
\usepackage{listings}
\usepackage[portuguese]{babel}
\usepackage{amsmath}
\usepackage{amssymb}

\begin{document}

\title{Questão 4 - Prova de otimização}
\author{Luiz Augusto Dembicki Fernandes, GRR20202416}
\date{\today}
\maketitle

\paragraph{A)}

\subparagraph{i)}
Para maximizar o lucro precisamos levar em conta os custos de cada carvão e o preço de venda assim temos:

\[ lucro \ com \ c = 2110  R\$ \cdot (x + y)  - (x \cdot 2000  R\$ + y \cdot 1500 R\$ ) \]

Onde x e y são quantidades de carvão A e B em toneladas.
Como queremos trabalhar com um problema de minimização, a função objetivo se torna então:
\[  - lucro \ com \ c = - 2110  R\$ \cdot (x + y)  + (x \cdot 2000  R\$ + y \cdot 1500 R\$ )  \]
As restrições especificadas:
\[ 92 \% \cdot x + 81 \% \cdot y \leqslant  88 \% (x + y) \]
\[ 1 \% \cdot x + 2 \% \cdot y \leqslant 1.5 \% (x + y) \]
\[ 7 \% \cdot x +  14 \% \cdot y <=  10 \% (x + y) \]
\[ x + y \geqslant 26 \%  \]

\subparagraph{ii)}

Foi utilizado python em ambiente jupyter notebook, com o método simplex modificado(HiGHS),da biblioteca Scipy, já que se trata de um problema de programação linear.
Tal método atua por operações nas matrizes, que geometricamente o problema se torna encontrar um vértice qual seja o ponto ótimo, por meio de operações
lineares é possível mover de vértice em vértice.

Foram criados dois casos, um em caso de toneladas forem variáveis continuas resultando em A: 16.54545 ton; B: 9.45455 ton; Lucro: 7587.27273 R\$;
E para variáveis discretas, neste caso inteiras, A: 17 ton; B: 9 ton; Lucro: 7360 R\$
Segue o código fonte:
\begin{lstlisting}[language=Python]

from scipy.optimize import linprog

# a) 
# Digitos GRR
X = 6
Y = 4
Z = 1

PC = 2100 + (Z * 10) # Preco de venda de C por tonelada em reais
# Funcao objetivo
Fobj = (- PC + 2000, - PC + 1500) # Custos e preco de venda
# Inequacoes
# lado direito
LD = [[-0.04, 0.07], # inequacao 1 Carbono -4% * x + 7% * y <= 0 
    [-0.005, 0.005], # inequacao 2 Enxofre -0.5% * x + 0.5% * y <= 0
    [-0.03, Y * 1e-2], # inequacao 3 cinzas -3% * x + Y% * y <= 0
    [1, 1]] # inequacao 4 demanda - x - y <= - 20 - X  
# lado esquerdo
LE = [0, # inequacao 1 Carbono -4% * x + 7% * y <= 0 
    0, # inequacao 2 Enxofre -0.5% * x + 0.5% * y <= 0 
    0, # inequacao 3 cinzas -3% * x + Y% * y <= 0
    20 + X] # inequacao 4 demanda  - x - y <= - 20 - X
# Optimizacao por Simplex ++ 
opt = linprog(c = Fobj, A_ub = LD, b_ub = LE)
print(f"A:{opt.x[0]:.5f} B:{opt.x[1]:.5f}, Lucro: {opt.fun:.5f}")
opt = linprog(c = Fobj, A_ub = LD, b_ub = LE, integrality = 1)
print(f"A:{opt.x[0]} B:{opt.x[1]}, Lucro: {opt.fun}")
\end{lstlisting}

\paragraph{B)}
Tal qual anterior a função objetivo é dependente do custo de compra e de venda:
\[ A + B + C - (C1 + C2 + C3 + C4) \]
como queremos em termos de minimização:
\[ F_{obj} = - ( A + B + C ) + C1 + C2 + C3 + C4 \]
E teremos as seguintes restrições, advindas dos blends requeridos, da conservação da massa e da quantidade máxima de barris :
\[ C1 \leqslant  15 \% A \]
\[ C2 \geqslant 40\% A \]
\[ C3 \leqslant  50 \% A \]
\[ C1 \leqslant 10\% B \]
\[ C2 \geqslant 10\% B \]
\[ C1 \leqslant 20\% C \]
\[ C1 + C2 + C3 + C4 = A + B + C \]
\[ C1 \leqslant 3000 \]
\[ C2 \leqslant 2000 \]
\[ C3 \leqslant 4000 \]
\[ C4 \leqslant 1000 \]
Também foi assumido que só é possível adquirir e comercializar um barril inteiro:
\[ \{C1, C2, C3, C4, A, B, C\} \in \mathbb{Z} \]

Com isso foi também utilizado um método simplex modificado, dessa vez com o pacote pulp, obtendo
C1 = 121 barris,
C2 = 2000 barris,
C3 = 2500 barris,
C4 = 1000 barris, lucro total =  6935.5 \$ / dia

Segue o código fonte:
\begin{lstlisting}[language=Python]
    import pulp
    # b)
    # problema
    lucro = pulp.LpProblem("lucro", pulp.LpMinimize)
    # Variaveis
    C1 = pulp.LpVariable("C1", 0, 3000, pulp.LpInteger)
    C2 = pulp.LpVariable("C2", 0, 2000, pulp.LpInteger)
    C3 = pulp.LpVariable("C3", 0, 4000, pulp.LpInteger)
    C4 = pulp.LpVariable("C4", 0, 1000, pulp.LpInteger)
    A = pulp.LpVariable("A", 0, cat = pulp.LpInteger)
    B = pulp.LpVariable("B", 0, cat = pulp.LpInteger)
    C = pulp.LpVariable("C", 0, cat = pulp.LpInteger)
    # equacoes e inequacoes
    lucro += 13 * C1 + 15.3 * C2 + 14.6 * C3 + 14.9 * C4 - (16.2 * A + 15.75 * B + 15.3 * C), "lucro total" # funcao objetivo
    # restricoes
    lucro += ( C1 - 15 * 1e-2 * A <= 0 , "Blend C1 em A")
    lucro += ( - C2  + 40 * 1e-2 * A <= 0 , "Blend C2 em A")
    lucro += ( C3 - 50 * 1e-2 * A <= 0 , "Blend C3 em A")
    lucro += ( C1 - 10 * B <= 0 , "Blend C1 em B")
    lucro += ( - C2 + 10 * 1e-2 * B <= 0 , "Blend C2 em B")
    lucro += ( C1 - 20 * 1e-2 * C <= 0 , "Blend C1 em C")
    lucro += (C1 + C2 + C3 + C4 - (A + B + C) == 0, "conservacao de massa")
    
    lucro.writeLP("Questao4b")
    lucro.solve()
    print("Status:", pulp.LpStatus[lucro.status])
    for v in lucro.variables():
        print(v.name, "=", v.varValue)
    print("lucro total = ", - pulp.value(lucro.objective))
\end{lstlisting}

\end{document}