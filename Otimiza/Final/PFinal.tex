\documentclass[]{article} % fonte 12, frente e verso, a4
\usepackage{xcolor}
\usepackage{listings}
\usepackage[portuguese]{babel}
\usepackage{amsmath}
\usepackage{amssymb}
\usepackage{geometry}
 \geometry{
 a4paper,
 total={170mm,257mm},
 left=20mm,
 top=20mm,
 }

\begin{document}

\title{Prova Final - Otimização de Processos}
\author{Luiz Augusto Dembicki Fernandes, GRR20202416}
\date{\today}
\maketitle

\paragraph{Questão 1}

\paragraph{a)} Verdadeira.

\paragraph{b)} Falsa, R1), R2), R3) e R4) são verdadeiras, no entanto R5) deveria ser
``O fertilizante B deve conter pelo menos 50\% de amônia", sua equação está correta.
\paragraph{c)} Falsa, vide b)
\paragraph{d)} Falsa, vide b)
\paragraph{e)} Falsa, somente C1 é limitado.
\paragraph{f)} Verdadeira
\paragraph{g)} Falsa
\paragraph{h)} Verdadeira
\paragraph{i)} Falsa, ambas são convexas.
\paragraph{j)} Falsa, o ponto quebra a restrição de disponibilidade do componente C1.
\paragraph{k)} Verdadeira, é obrigatório realizar a criação de formas canônicas e então
pivotear para encontrar a solução adequada.

\paragraph{Questão 2}

\paragraph{1)} Falsa, na b) e d) não coincidem o ótimo com restrição e o ótimo das funções
\paragraph{2)} Falsa, se refere a um mínimo global
\paragraph{3)} Falsa, na figura A o ponto ótimo não é modificado com as restrições
\paragraph{4)} Falsa, o ponto A seria ótimo se não houvessem restrições.
\paragraph{5)} Falsa, método de Newton tem convergência quadratica, seriam necessários
mais passos que um para todas.
\paragraph{6)} Verdadeira.
\paragraph{7)} Verdadeira.
\paragraph{8)} Falsa, na figura D possui 5 restrições.
\paragraph{9)} Verdadeira, mas não é ponto de ótimo.
\paragraph{10)} Falsa, o ótimo global é representado por \(x_{opt}^{(2)}\).

\end{document}