\documentclass[]{article} % fonte 12, frente e verso, a4
\usepackage{xcolor}
\usepackage{listings}
\usepackage[portuguese]{babel}
\usepackage{amsmath}
\usepackage{amssymb}
\usepackage{geometry}
 \geometry{
 a4paper,
 total={170mm,257mm},
 left=20mm,
 top=20mm,
 }

\begin{document}

\title{Prova 2 - Otimização de Processos}
\author{Luiz Augusto Dembicki Fernandes, GRR20202416}
\date{\today}
\maketitle

\paragraph{Questão 1}

Trata-se de um problema de minimização de custos com a equação de custo dependente da
área, assim utilizando as equações:
\[ Q = U \cdot A \cdot \Delta T_{lm} \]
\[ T_{lm} = \frac{(T_1 - t_2) - (T_2 - t_1)}{ln  \frac{T_1 - t_2}{T_2 - t_1}}  \]
\[ Q = m \cdot C_p \cdot \Delta T \]

As área foi obtida da seguinte forma, supondo que não há perda de
energia:
\[ Q =  U \cdot A \cdot \Delta T_{lm} = m \cdot C_p \cdot \Delta T \]
\[ \to A = \frac{ m \cdot C_p \cdot \Delta T}{U \cdot \Delta T_{lm}} \]
Assim para o caso em série:
\[ Q1 \to Q = 16670 \cdot (252 - 320) = - 14450 (t_2 - 140) \therefore t_2 = 218,45 ^oF \]
\[ \therefore A_2 = \frac{ 1133560 }{ 106 \cdot 106,7} = 100,22 \ ft^2 \]
\[ A_3 = \frac{ 20000 \cdot (353 - 280)}{ 106 \cdot 45,8 } = 300,73 \ ft^2 \]
Utilizando os valores encontrados na função objetivo temos então um custo fixo para
o esquema em série:
\[ C_{cap} = 100,22^{0,36} + 300,73^{0,21} = 8,57 \ \$ \slash A \]

Para o esquema em paralelo foi variado a composição por meio da substituição da fração
no lugar de \(m\). Por meio do solver do Excel variou-se x de modo a mínimizar \(C_{cap}\),
com \(x\) inicial de 0,5 já que este oferecia resultados congruentes para \(\Delta T_{lm}\).
Solver convergiu em \(x \approx 0,5\) com um \(C_{cap} = 7,98 \ \$ \slash A \), ou seja
é mais avantajoso o processo em paralelo.
Equações para paralelo:
\begin{center}
    min \( C_{cap} = A_2 ^{0,36} + A_3 ^{0,21} \), sujeito à:
    \[ A = \frac{ x \cdot m \cdot C_p \cdot \Delta T}{U \cdot \Delta T_{lm}} \ \vert \ x \in \mathbb{R} ; 0,0001 \leqslant x \leqslant 1\]
    \[ T_{lm} = \frac{(T_1 - t_2) - (T_2 - t_1)}{ln  \frac{T_1 - t_2}{T_2 - t_1}} \]
    \[ t_2 = \frac{Q}{ x \cdot m \cdot C_p} + 140\]
\end{center}

\paragraph{Questão 2}
a) Utilizando as relações:
\[ Q - W_0 C_p (t_{i-1}-t_i) = 0\]
\[ Q - \lambda W_i = 0 \]
\[\to Q - W_0 C_p (t_{i-1}-t_i) = Q - \lambda W_i \therefore W_i = \frac{W_0 C_p (t_{i-1}-t_i)}{\lambda}\]
Desta forma:
\[ W_1 = 3465,3 \ lb \ ; \ W_2 = 3663,3 \ lb \ ; \ W_3 = 4752,5 \ lb \]
\[ A = \frac{ Q }{U \cdot \Delta T_{lm}} \]
\[ \to A_1 = 58,4 \ ft^2 \ ; \ A_2 = 54,2 \ ft^2 \ ; \ A_3 = \ 85,3 ft^2 \ \]

b) Com Excel foram adicionados as relações acima em celulas e otimizadas com solver, primeiramente um a um
cada trocador foi dimensionado variando a temperatura, no entanto desta forma o resultado será sempre a temperatura
mais próxima da inicial possível, portanto não é uma forma viável de resolução. No entanto para otimização de ambos
ao mesmo tempo, foi utilizada a função objetivo o custo total, minimizando-o, solver convergiu para \( t_1 = 49.9 ^o F\)
e \( t_2 = 49.8 ^o F\) que estão no limiar do dominio fornecido, demonstrando que o sistema converge para maximizar a capacidade
do ultimo trocador em detrimento do resto. A região viavel é composta da temperatua menor que a anterior mas maior que a posterior,
e maior também que a temperatura do refrigerante.

c) Custo é mais sensivel para com \( U \) já que este influencia diretamente na área que é variável na função custo,
com o aumento de \(U\) reduz-se o custo e vice-versa.

\paragraph{Questão 3}

Modelagem:
min \( - Lucro = - 0,4 \cdot (AB_{e1} + AB_{e2}) + 0,01 \cdot (W_{1} + W_2) \) sujeito à:
\[ \frac{AB_{ei}}{W_{i} + AB_{ie}} = y = 4 \cdot x_i \]
\[ AB_{ei} = ABA_i - ABA_{i-1} \]
\[ ABA_0 = 102,125 \cdot 0,026 = 2,65525 \]
\[ \sum ( AB_{ei} + ABA_i ) = 2,65525 \]
\[ x_i \leqslant x_{i - 1} \]
\[ {x_i, y_i} \in [0,1]  \]
Onde i é o onde se encontra no processo (0: antes da extração, 1: extração 1
2: extração 2) Como não foi fornecido \(x\) utilizou-se \(ABA_1\) e \(ABA_2\) como variaveis junto com
\(W_1\) e \(W_2\). Com solver foi encontrado:
\[ W_1 = 22,88 \ kg \ ; W_2 = 2 \ kg \]
E lucro de 0,81 \$, com chute inicial de 10 e 1 kg de \(W_1\) e \(W_2\) respectivamente.

\end{document}